\documentclass[]{article}
\usepackage[utf8]{inputenc}
\usepackage[russian]{babel}
\usepackage{graphicx}
\usepackage{pgfplots}
\usepackage[papersize={21cm,29.7cm}]{geometry}
\author{Овчинников Никита 7 группа}
\begin{document}

\begin{titlepage}
\maketitle 
\title Лабораторная работа № 1 «Численное решение нелинейных уравнений»
\end{titlepage}

\begin{center}
	\huge\textbf{Постановка задачи}
\end{center}

Условие:

~

$x^5 - 7*x^2 + 3 = 0$

\newpage
\begin{center}
	\huge\textbf{Решение}
\end{center}

\begin{center}
\begin{tikzpicture}
\begin{axis}[
	xlabel = {$x$},
	ylabel = {$y$},
	width=12cm, height=12cm,
	xmin = -1.5, xmax = 2.5,
	minor tick num = 1,
	legend pos = north west,
	samples = 1000
]
\legend{
	$x^5 - 7 * x^2 + 3$,
	$7*x^2 - 3$,
	$x^5$
};
\draw[solid, help lines] (axis cs:-5,0) -- (axis cs:5,0);
\draw[solid, help lines] (axis cs:0,-500) -- (axis cs:0,500);
\addplot[solid, draw = black] {x^5 - 7 * x^2 + 3};
\addplot[solid, draw = red] {7 * x^2 - 3};
\addplot[solid, draw = green] {x^5};
\end{axis}
\end{tikzpicture}
\end{center}

Отделим корни: [-1, 0], [0, 1], [1, 2]

\newpage

\begin{center}
	\huge\textbf{Дехотомия}
\end{center}

Для [-1, 0]:
\begin{center}
\begin{tabular}{ |r|r r r r r r r| }
\hline
k & $a_k$ & $b_k$ & $mid_k$ & $f(l_k)$ & $f(r_k)$ & $f(mid_k)$ & $r_k-l_k$\\ 	
\hline
0 & -1.000 & 0.000 & -0.500 & -5.000 & 3.000 & 1.219 & 1.000 \\
1 & -1.000 & -0.500 & -0.750 & -5.000 & 1.219 & -1.175 & 0.500 \\
2 & -0.750 & -0.500 & -0.625 & -1.175 & 1.219 & 0.170 & 0.250 \\
3 & -0.750 & -0.625 & -0.688 & -1.175 & 0.170 & -0.462 & 0.125 \\
4 & -0.688 & -0.625 & -0.656 & -0.462 & 0.170 & -0.136 & 0.062 \\
5 & -0.656 & -0.625 & -0.641 & -0.136 & 0.170 & 0.019 & 0.031 \\
6 & -0.656 & -0.641 & -0.648 & -0.136 & 0.019 & -0.058 & 0.016 \\
7 & -0.648 & -0.641 & -0.645 & -0.058 & 0.019 & -0.019 & 0.008 \\
8 & -0.645 & -0.641 & -0.643 & -0.019 & 0.019 & 0.000 & 0.004 \\
\hline
\end{tabular}
\end{center}

Для [0, 1]:
\begin{center}
\begin{tabular}{ |r|r r r r r r r| }
\hline
k & $a_k$ & $b_k$ & $mid_k$ & $f(l_k)$ & $f(r_k)$ & $f(mid_k)$ & $r_k-l_k$\\
\hline
0 & 0.000 & 1.000 & 0.500 & 3.000 & -3.000 & 1.281 & 1.000 \\
1 & 0.500 & 1.000 & 0.750 & 1.281 & -3.000 & -0.700 & 0.500 \\
2 & 0.500 & 0.750 & 0.625 & 1.281 & -0.700 & 0.361 & 0.250 \\
3 & 0.625 & 0.750 & 0.688 & 0.361 & -0.700 & -0.155 & 0.125 \\
4 & 0.625 & 0.688 & 0.656 & 0.361 & -0.155 & 0.107 & 0.062 \\
5 & 0.656 & 0.688 & 0.672 & 0.107 & -0.155 & -0.023 & 0.031 \\
6 & 0.656 & 0.672 & 0.664 & 0.107 & -0.023 & 0.042 & 0.016 \\
7 & 0.664 & 0.672 & 0.668 & 0.042 & -0.023 & 0.010 & 0.008 \\
\hline
\end{tabular}
\end{center}

Для [1, 2]:
\begin{center}
\begin{tabular}{ |r|r r r r r r r| }
\hline
k & $a_k$ & $b_k$ & $mid_k$ & $f(l_k)$ & $f(r_k)$ & $f(mid_k)$ & $r_k-l_k$\\
\hline
0 & 1.000 & 2.000 & 1.500 & -3.000 & 7.000 & -5.156 & 1.000 \\
1 & 1.500 & 2.000 & 1.750 & -5.156 & 7.000 & -2.024 & 0.500 \\
2 & 1.750 & 2.000 & 1.875 & -2.024 & 7.000 & 1.565 & 0.250 \\
3 & 1.750 & 1.875 & 1.812 & -2.024 & 1.565 & -0.435 & 0.125 \\
4 & 1.812 & 1.875 & 1.844 & -0.435 & 1.565 & 0.511 & 0.062 \\
5 & 1.812 & 1.844 & 1.828 & -0.435 & 0.511 & 0.024 & 0.031 \\
6 & 1.812 & 1.828 & 1.820 & -0.435 & 0.024 & -0.209 & 0.016 \\
7 & 1.820 & 1.828 & 1.824 & -0.209 & 0.024 & -0.093 & 0.008 \\
8 & 1.824 & 1.828 & 1.826 & -0.093 & 0.024 & -0.034 & 0.004 \\
9 & 1.826 & 1.828 & 1.827 & -0.034 & 0.024 & -0.005 & 0.002 \\
\hline
\end{tabular}
\end{center}

\newpage
\begin{center}
	\huge\textbf{Метод простой итерации}
\end{center}

Приведем функцию к виду:

$x = \sqrt{\frac{x^5 + 3}{7}}$
и
$x = (7*x^2 -3)^{\frac{1}{5}}$

~

\begin{center}
\begin{tikzpicture}
\begin{axis}[
	xlabel = {$x$},
	ylabel = {$y$},
	width=12cm, height=7cm,
	xmin = -2, xmax = 3,
	ymin = -1.5, ymax = 4.5,
	minor tick num = 1,
	legend pos = south east,
	samples = 1000
]
\legend{
	$\sqrt{\frac{x^5 + 3}{7}}$,
	$(7*x^2 -3)^{\frac{1}{5}}$
};
\draw[solid, help lines] (axis cs:-5,0) -- (axis cs:5,0);
\draw[solid, help lines] (axis cs:0,-5) -- (axis cs:0,5);
\addplot[solid, draw = blue, domain=(-3^(1/5)):3] {sqrt((x^5 + 3)/7)};
\addplot[solid, draw = red, domain=(-2):sqrt(3)/(-7)] {(7*x^2 - 3)^(1/5)};
\addplot[solid, draw = red, domain=sqrt(3)/(7):3] {(7*x^2 - 3)^(1/5)};
\end{axis}
\end{tikzpicture}
\end{center}

Исследуем на сходимость:

\begin{center}
\begin{tikzpicture}
\begin{axis}[
	xlabel = {$x$},
	ylabel = {$y$},
	width=12cm, height=8cm,
	xmin = -2, xmax = 3,
	ymin = -2, ymax = 5,
	minor tick num = 1,
	legend pos = south east,
	samples = 1000
]
\legend{
	$(\sqrt{\frac{x^5 + 3}{7}})'$,
	$((7*x^2 -3)^{\frac{1}{5}})'$
};
\draw[solid, help lines] (axis cs:-5,0) -- (axis cs:5,0);
\draw[solid, help lines] (axis cs:0,-5) -- (axis cs:0,5);
\draw[dashed, help lines] (axis cs:-5,1) -- (axis cs:5,1);
\draw[dashed, help lines] (axis cs:-1,-5) -- (axis cs:-1,5);
\draw[dashed, help lines] (axis cs:0,-5) -- (axis cs:0,5);
\draw[dashed, help lines] (axis cs:1,-5) -- (axis cs:1,5);
\draw[dashed, help lines] (axis cs:2,-5) -- (axis cs:2,5);
\addplot[solid, draw = blue] {1/2 * (1/sqrt((x^5 + 3) / 7)) * (5 * x^4)/7};
\addplot[solid, draw = red, domain=0.5:5] {1/5 * 7*2*x * (7*x^2 - 3)^(-4/5)};
\addplot[solid, draw = red, domain=-5:0.5] {1/5 * 7*2*x * (7*x^2 - 3)^(-4/5)};
\end{axis}
\end{tikzpicture}
\end{center}

Условие сходимости :  $|\varphi'(x)| < 1$

~

Итерационный процесс:

$x_{k+1} = \varphi(x_{k}) * sign(x_{k})$

~

Как можно заметить, 
$x = \sqrt{\frac{x^5 + 3}{7}}$
будет сходиться на отрезке [-1, 1], а 
$x = (7*x^2 -3)^{\frac{1}{5}}$
на отрезке  [1, 2].

~

\begin{tabular}[t]{ |r|r r| }
\hline
k & $x_k$ & $\epsilon_k$\\
\hline
1 & -0.6512351 & 0.1512351 \\
2 & -0.6417459 & 0.0094892 \\
3 & -0.6426678 & 0.0009219 \\
4 & -0.6425806 & 0.0000872 \\
5 & -0.6425889 & 0.0000083 \\
6 & -0.6425881 & 0.0000008 \\
7 & -0.6425882 & 0.0000001 \\
\hline
\end{tabular}
\begin{tabular}[t]{ |r|r r| }
\hline
k & $x_k$ & $\epsilon_k$\\
\hline
1 & 0.6580545 & 0.1580545 \\
2 & 0.6679819 & 0.0099274 \\
3 & 0.6690069 & 0.0010250 \\
4 & 0.6691161 & 0.0001093 \\
5 & 0.6691278 & 0.0000117 \\
6 & 0.6691291 & 0.0000013 \\
7 & 0.6691292 & 0.0000001 \\
8 & 0.6691292 & 0.0000000 \\
\hline
\end{tabular}
\begin{tabular}[t]{ |r|r r| }
\hline
k & $x_k$ & $\epsilon_k$\\
\hline
1 & 1.6638035 & 0.1638035 \\
2 & 1.7492447 & 0.0854411 \\
3 & 1.7908252 & 0.0415806 \\
4 & 1.8104276 & 0.0196024 \\
5 & 1.8195351 & 0.0091075 \\
6 & 1.8237383 & 0.0042032 \\
7 & 1.8256722 & 0.0019339 \\
8 & 1.8265607 & 0.0008885 \\
9 & 1.8269687 & 0.0004080 \\
10 & 1.8271559 & 0.0001873 \\
11 & 1.8272419 & 0.0000859 \\
12 & 1.8272813 & 0.0000394 \\
13 & 1.8272994 & 0.0000181 \\
14 & 1.8273077 & 0.0000083 \\
15 & 1.8273115 & 0.0000038 \\
16 & 1.8273133 & 0.0000017 \\
17 & 1.8273141 & 0.0000008 \\
18 & 1.8273144 & 0.0000004 \\
19 & 1.8273146 & 0.0000002 \\
20 & 1.8273147 & 0.0000001 \\
\hline
\end{tabular}


\end{document}
